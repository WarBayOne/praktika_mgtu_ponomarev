%------------------Settings-------------------------

\documentclass[12pt]{article}
\usepackage[utf8]{inputenc}
\usepackage[russian]{babel}
\usepackage{amsmath,amssymb}
\usepackage{graphics}
\usepackage{pbox}
\usepackage[x11names]{xcolor}
\definecolor{brightmaroon}{rgb}{0.76, 0.13, 0.28}
\definecolor{royalazure}{rgb}{0.0, 0.22, 0.66}
\usepackage[colorlinks=true,linkcolor=royalazure]{hyperref}
\usepackage{tikz, tkz-fct, pgfplots}
\usetikzlibrary{arrows}
\usepackage{geometry}
\geometry{
	a4paper,
	total={170mm,257mm},
	left=20mm,
	top=20mm
} 
\usepackage[labelsep=period]{caption}

% ----------------- Commands ----------------- 

\newcommand{\eps}{\varepsilon}
\newcommand\tline[2]{$\underset{\text{#1}}{\text{\underline{\hspace{#2}}}}$}

% ----------------- Set graphics path ----------------- 
\graphicspath{{img/}}
\begin{document}
\pagestyle{empty}

% ----------------------Title----------------------------------
\centerline{\large Министерство науки и высшего образования}	
\centerline{\large Федеральное государственное бюджетное образовательное}
\centerline{\large учреждение высшего образования}
\centerline{\large ``Московский государственный технический университет}
\centerline{\large имени Н.Э. Баумана}
\centerline{\large (национальный исследовательский университет)''}
\centerline{\large (МГТУ им. Н.Э. Баумана)}
\hrule
\vspace{0.5cm}
\begin{figure}[h]
\center
\includegraphics[height=0.35\linewidth]{bmstu-logo-small.png}
\end{figure}
\begin{center}
	\large	
	\begin{tabular}{c}
		Факультет ``Фундаментальные науки'' \\
		Кафедра ``Высшая математика''		
	\end{tabular}
\end{center}
\vspace{0.5cm}
\begin{center}
	\LARGE \bf	
	\begin{tabular}{c}
		\textsc{Отчёт} \\
		по учебной практике \\
		за 1 семестр 2020---2021 гг.
	\end{tabular}
\end{center}
\vspace{0.5cm}
\begin{center}
	\large
	\begin{tabular}{p{5.3cm}ll}
		\pbox{5.45cm}{
			Руководитель практики,\\
			ст. преп. кафедры ФН1} 	& \tline{\it(подпись)}{5cm} & Кравченко О.В. \\[0.5cm]
		студент группы ФН1--11 		& \tline{\it(подпись)}{5cm} & Пономарёв М.А.
	\end{tabular}
\end{center}
\vfill
\begin{center}
	\large	
	\begin{tabular}{c}
		Москва, \\
		2020 г.
	\end{tabular}
\end{center}
 \newpage
 \newpage	
\tableofcontents

%------------------Table of contents----------------------
\newpage
\pagestyle{plain}
\setcounter{page}{3}
\section{Цели и задачи практики}	
\subsection{Цели}
--- развитие компетенций, способствующих успешному освоению материала бакалавриата и необходимых в будущей профессиональной деятельности.

\subsection{Задачи}
\begin{enumerate}
\item Знакомство с программными средствами, необходимыми в будущей профессиональной деятельности.
\item Развитие умения поиска необходимой информации в специальной литературе и других источниках.
\item Развитие навыков составления отчётов и презентации результатов.
\end{enumerate}

\subsection{Индивидуальное задание}	
\begin{enumerate}
\item Изучить способы отображения математической информации в системе вёртски \LaTeX.
\item Изучить возможности  системы контроля версий \textsf{Git}.
\item Научиться верстать математические тексты, содержащие формулы и графики в системе \LaTeX.
Для этого, выполнить установку свободно распространяемого дистрибутива \textsf{TeXLive} и оболочки \textsf{TeXStudio}.
\item Оформить в системе \LaTeX типовые расчёты по курсе математического анализа согласно своему варианту.
\item Создать аккаунт на онлайн ресурсе \textsf{GitHub} и загрузить исходные \textsf{tex}--файлы 
и результат компиляции в формате \textsf{pdf}.
\end{enumerate} 

%---------------------------------------------------------------
\newpage
\section{Отчёт}
Актуальность темы продиктована необходимостью владеть системой вёрстки \LaTeX и средой вёрстки \textsf{TeXStudio} для
отображения текста, формул и графиков. Полученные в ходе практики навыки могут быть применены при написании
курсовых проектов и дипломной работы, а также в дальнейшей профессиональной деятельности.

Ситема вёрстки \LaTeX содержит большое количество инструментов (пакетов), упрощающих отображение информации в различных 
сферах инженерной и научной деятельности. 
%-----------------------------------------------------------------
\newpage
\section{Индивидуальное задание}
%\subsection{Элементарные функции и их графики.}
%\input{src/part1.tex}

%==============================================================================
\subsection{Пределы и непрерывность.}

%---------------------------- Problem 1----------------------------------
\subsubsection*{\center Задача № 1.}
{\bf Условие.~}
Дана последовательность $a_{n}=\dfrac{5n+11}{6-n}$ и число $c=-5$. Доказать, что $\lim\limits_{x\rightarrow\infty} a_{n}=c $, а именно, для каждого $\varepsilon>0$ найти наименьшее натуральное число  $N{=}N(\varepsilon)$ такое, что $|a_{n}-c|<\varepsilon$ для всех $n>N(\varepsilon)$. Заполнить таблицу: 
\begin{center}
\begin{tabular}{ | p{25pt} | c | c | c | c |}
\hline
$\varepsilon$& $0{,}1$ & $0{,}01$ & $0{,}001$ \\ \hline
$N(\varepsilon)$ &   &   &\\
\hline
\end{tabular}
\end{center}
\medskip
%=====================================================================
{\bf Решение.~}
Рассмотрим неравенство $a_{n}-c<\varepsilon$, $\forall\varepsilon>0$, учитывая выражение для $a_{n}$ и $c$ из условия варианта, получим 
$$\left|\frac{5n+11}{6-n}+5\right|<\varepsilon.$$
Неравенство запишем в виде двойного неравенства и приведём выражение под знаком модуля к общему знаменателю, получим
$${-}\varepsilon <\dfrac{41}{n-6}<\varepsilon.$$
Заметим, что левое неравенство выполнено для любого номера $n\in \mathbb{N}$ поэтому, будем рассматривать правое неравенство
$$\frac{41}{n-6}<\varepsilon.$$
Выполнив цепочку преобразований, перепишем неравенство относительно $n$, и, учитывая, что $n\in \mathbb{N}$, получим 
$$\dfrac{41}{n-6}<\varepsilon,$$
$$n-6>\dfrac{41}{\varepsilon},$$
$$n>\dfrac{41}{\varepsilon}+6,$$
$$n>\dfrac{41+6\varepsilon}{\varepsilon},$$
$$N(\varepsilon)=\biggl[\dfrac{41+6\varepsilon}{\varepsilon}\biggr],$$
где $[\;]$ -- целая часть от числа. Заполним таблицу:
\begin{center}
\begin{tabular}{ | p{25pt} | c | c | c | c |}
\hline
$\varepsilon$& $0{,}1$ & $0{,}01$ & $0{,}001$ \\ \hline
$N(\varepsilon)$ & 417  & 4107 & 41007\\
\hline
\end{tabular}
\end{center}
{\bf Проверка:~}
$$|a_{418}-c|=\dfrac{41}{412}<0{,}1,$$
$$|a_{4108}-c|=\dfrac{41}{4102}<0{,}01,$$
$$|a_{41008}-c|=\dfrac{41}{41002}<0{,}001.$$
\newpage
% ---------------------------- Problem 2----------------------------------
\subsubsection*{\center Задача № 2.}
{\bf Условие.~}
Вычислить пределы функций:
$$
\begin{center}
\begin{array}{cc}
\text{\bf(а):} &  \lim\limits_{x\rightarrow2}\dfrac{x^3-5x+2}{x^2-12x+20} , \\[10pt]
\text{\bf(б):} & \lim\limits_{x\rightarrow+\infty} \dfrac{x^3+\sqrt{x^8+9x^5}+x^4}{3x^4-2x^3} ,\\[10pt]
\text{\bf(в):} & \lim\limits_{x\rightarrow1} \dfrac{\sqrt[4]{x}-1}{\sqrt[3]{x}-1},\\[10pt]
\text{\bf(г):} & \lim\limits_{x\rightarrow \frac{3}{2}} \biggl( 2-\dfrac{2x}{3}\biggl)^{ \tg \frac{\pi x}{3}}, \\[10pt]
\text{\bf(д):} & \lim\limits_{x\rightarrow0} \left(\dfrac{\sin{2x}}{\lg(x+1)}\right)^{\frac{2}{x+\cos x}} , \\[10pt]
\text{\bf(е):}  & \lim\limits_{x\rightarrow0} \dfrac{1-\cos{5x}}{\varepsilon^{x^2}-1} . \\
\end{array}
\end{center}

$$
\begin{flushleft}
{\bf Решение.~}
\medskip

\text{\bf(а):} 
$$

\lim\limits_{x\rightarrow2}\dfrac{x^3-5x+2}{x^2-12x+20} =  \lim\limits_{x\rightarrow2}  \dfrac{(x-2)(x^2+2x-1)}{(x-2)(x-10)} = \lim\limits_{x\rightarrow-1}  \dfrac{x^2+2x-1}{x-10}=-\dfrac{7}{8}.

$$

\text{\bf(б):}
$$


\lim\limits_{x\rightarrow+\infty} \dfrac{x^3+\sqrt{x^8+9x^5}+x^4}{3x^4-2x^3}=\lim\limits_{x\rightarrow+\infty} \dfrac{\dfrac{1}{x}+\sqrt{1+\dfrac{9}{x^3}}+1}{3-\dfrac{2}{x}}=\dfrac{2}{3}.


$$

\text{\bf(в):}
 $$
 
 \lim\limits_{x\rightarrow1} \dfrac{\sqrt[4]{x}-1}{\sqrt[3]{x}-1}=\lim\limits_{x\rightarrow1} \dfrac{x-x^4}{x^{\frac{4}{3}}-x^4}=\lim\limits_{x\rightarrow1} \dfrac{x-1}{x^{\frac{4}{3}}-1}=\lim\limits_{x\rightarrow1}\dfrac{(x-1)(1+x^{\frac{4}{3}}+x^{\frac{8}{3}})}{(x-1)(x^3+x^2+x+1)}=\lim\limits_{x\rightarrow1}\dfrac{1+x^{\frac{4}{3}}+x^{\frac{8}{3}}}{x^3+x^2+x+1}=\dfrac{3}{4}.
 
 $$

\text{\bf(г):}
$$

\lim\limits_{x\rightarrow \frac{3}{2}} \biggl( 2-\dfrac{2x}{3}\biggl)^{ \tg \frac{\pi x}{3}}= \biggr| \begin{array}{l}
t=2x/3} \\ t\rightarrow 1
\end{array}
\biggr|=\lim\limits_{t\rightarrow 1} \biggl( 2-t\biggl)^{ \tg \frac{\pi t}{2}}=\biggr| \begin{array}{l}
y=-t-1} \\ t=-y-1
\end{array}
\biggr|=\lim\limits_{y\rightarrow 0} 3^{ \tg \frac{\pi(1-y)}{2}}=\lim\limits_{y\rightarrow 0} 3^{ \ctg \frac{\pi y}{2}}=\exp\biggl ( \dfrac{2}{\pi}\lim\limits_{y\rightarrow 0} \cos \dfrac{\pi y}{2} \biggr )=\exp \biggl (\dfrac{2}{\pi} \biggr ) .

$$

\text{\bf(д):}
$$

\lim\limits_{x\rightarrow0} \left(\dfrac{\sin{2x}}{\lg(x+1)}\right)^{\frac{2}{x+\cos x}}=\lim\limits_{x\rightarrow0} \left(\dfrac{\sin{2x}\ln 10}{\ln(x+1)}\right)^{\frac{2}{x+\cos x}}=\lim\limits_{x\rightarrow0} \ln 10 ^ {\frac{2}{x+\cos x}} \lim\limits_{x\rightarrow0} \biggl ( \dfrac {\sin 2x}{\ln (x+1)} \biggr )^ {\frac{2}{x+\cos x}} = \ln 10 ^2 \biggl ( \lim \limits_{x\rightarrow0} \biggl( \dfrac{\sin2x}{\ln (x+1)} \biggr) \biggr)^2= 4 \ln ^2 10.

$$

\text{\bf(е):}
$$

  \lim\limits_{x\rightarrow0} \dfrac{1-\cos{5x}}{\varepsilon^{x^2}-1}= & \lim\limits_{x\rightarrow0} \dfrac{25x^2}{2x^2}=\dfrac{25}{2}.

\end{flushleft}
$$
% ---------------------------- Problem 3----------------------------------
\subsubsection*{\center Задача № 3.}
{\bf Условие.~}\\
\text{\bf(а):} Показать, что данные функции
$f(x)$ и $g(x)$ являются бесконечно малыми или бесконечно большими
при указанном стремлении аргумента. \\
\text{\bf(б):} Для каждой функции $f(x)$ и $g(x)$ записать главную часть
(эквивалентную ей функцию)  вида $C(x-x_0)^{\alpha}$ при $x\rightarrow x_0$ или $Cx^{\alpha}$
при $x\rightarrow\infty$, указать их порядки малости (роста). \\
\text{\bf(в):} Сравнить функции $f(x)$ и $g(x)$ при указанном стремлении.
\begin{center}
	\begin{tabular}{|c|c|c|}
		\hline
		№ варианта & функции $f(x)$ и $g(x)$ & стремление \\[6pt]
		\hline
		18 & $f(x) = \sin \dfrac {1}{\sqrt{x+\sqrt{x}+1}},~g(x)=\sqrt{x^2+\sqrt{x}}-x$ & $x\rightarrow+\infty$ \\
		\hline
		\end{tabular}
		\bigskip
		\\
		{\bf Решение.~}\\
		\end{center}
		\medskip
		\text{\bf(а):}~Покажем, что $f(x)$ и $g(x)$ бесконечно малые функции,
$$
 \begin{array}{l} 
\lim\limits_{x\rightarrow+ \infty} f(x)=\lim\limits_{x\rightarrow +\infty} \sin \dfrac {1}{\sqrt{x+\sqrt{x}+1}}=0, \\
 
\lim\limits_{x\rightarrow+ \infty} g(x)= \lim\limits_{x\rightarrow \infty} \biggl ( \sqrt{x^2+\sqrt{x}}-x \biggr ) =\lim\limits_{x\rightarrow+ \infty} \dfrac{\sqrt{x}+x^2-x^2}{\sqrt{x^2+\sqrt{x}}+x}}=\lim\limits_{x\rightarrow+ \infty}\dfrac{\sqrt{x}}{\sqrt{x^2+\sqrt{x}}+x}}=0.
\end{array}
$$
\text{\bf(б):}~Так как $f(x)$ и $g(x)$ бесконечно малые функции, то эквивалентными им будут функции вида 
$C(x-x_0)^{\alpha}$ при $x\rightarrow+\infty$.
Рассмотрим предел
$$
\lim\limits_{x\rightarrow +\infty} \dfrac{x^k}{ \sqrt{x+\sqrt{x}+1} }=\lim\limits_{x\rightarrow +\infty} \dfrac{x^{2k}}{ x+\sqrt{x}+1 }.
$$
При $k=\dfrac{1}{2}$ последний предел равен $1$, отсюда
$$
f(x)\sim \biggl( \dfrac{1}{x} \biggr)^{\frac{1}{2}} = x^{-\frac{1}{2}}  \text{ при}~x\rightarrow+\infty.
$$
Аналогично, рассмотрим предел
$$
\lim\limits_{x\rightarrow+\infty}\dfrac{x^k}{\sqrt{x+\frac{1}{\sqrt{x}}}+\sqrt{x}}
$$
При $k=\dfrac{1}{2}$ последний предел равен $\dfrac{1}{2}$, отсюда
$$
g(x)\sim\frac{1}{2} \biggl (\dfrac{1}{x} \biggr)^\frac{1}{2} = \dfrac{1}{2} x ^ {-\frac{1}{2}} \text{ при}~x\rightarrow+\infty.
$$
\text{\bf(в):}~Для сравнения функций $f(x)$ и $g(x)$ рассмотрим предел их отношения при указанном стремлении
$$
\lim\limits_{x\rightarrow\infty}\dfrac{f(x)}{g(x)}.
$$
Применим эквивалентности, определенные в пункте (б), получим
$$
\lim\limits_{x\rightarrow+\infty}\dfrac{f(x)}{g(x)} = \lim\limits_{x\rightarrow+\infty}\dfrac{2x^{-\frac{1}{2}}}{x^{-\frac{1}{2}}}=2.
$$
Отсюда, $g(x)$ и $f(x)$ - функции одинакового порядка малости.
%=================================================================================================================================
%\subsection{Приложения дифференциального исчисления.}
%\input{src/part3.tex}

\newpage
\addcontentsline{toc}{section}{Список литературы}
\begin{thebibliography}{99}
\bibitem{book01} Львовский С.М. Набор и вёрстка в системе \LaTeX, 2003 c.
\bibitem{book02} Е.М. Балдин Компьютерная типография \LaTeX.

\end{thebibliography}
\end{document}
